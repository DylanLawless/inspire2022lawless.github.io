\subsection{Other notes}
\begin{description}[noitemsep]
\item What Qs does the reader have?
\item Why is the variant present at low level in population. 
\item Chronic disease.
\item Airway reprogramming.
\item Cleared virus may have less of influence than chronic.
\item Infants without RSV less likely to have asthma.
\item Infants infected go on to have blunted subsequent antiviral responses.
\item Chronic stimulation versus immune exhaustion?
\item Metabolism of airway epithelium, glycolytic pathways.
\item How would this mutation lead to persistence? 
	- Epitope
	- Evasion
	- etc.
\item Selected in some backgrounds but not very fit?
\item Not increasing over time. Stable.
\item Heather Zar - papers
\item Acute and chronic resp morbidity to give it the spin for CID.
	\item Note that not only the persistent have variants of interest, but many others also have this variants. 
\end{description}
* Reservoirs within host? environmental? closer equator less seasonality. virus may retreat to parts of the world where it can overseason; temp, humidity, etc.
* Severity of second infection, Characterised URI/LRI and score
* Emergence?
* Kenya - family sampling every 5 days (tropical medicine funded this, and South Africa Heather Zar)

Enrolled in INSPIRE throughout infancy (i.e., the first year of life), 

Filter 1:
we conducted passive and active surveillance during their first RSV season by 
1)	- performing bi-weekly phone, 
	- email, and/or 
	- in person follow-up, 

2) frequently educating and reminding parents to call us at the onset of any acute respiratory symptoms, and 

3) approaching all infants who were seen at one of the participating pediatric practices for an unscheduled visit. 

Filter 2:
If an infant met pre-specified criteria for an acute respiratory infection, 
we then conducted an in-person respiratory illness visit at which time we:
	- administered a parental questionnaire, 
	- performed a physical exam, 
	- collected a nasal wash, 
	- and (in infants seen during an unscheduled visit) completed a structured medical chart review.

Filter 3: 
	- Nasal sample collections were assessed by reverse transcription-quantitative PCR for RSV [Jim to provide reference].

Filter 4:
	- At one year of age infants underwent blood draw for RSV serology to determine infection status during infancy. 

Filter 5:
Infants with positive PCR separated by more than 15 or more days were annotated during analysis as "persistent or repeat infection".